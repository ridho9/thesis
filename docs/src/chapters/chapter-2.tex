\chapter{Tinjauan Pustaka}

Pada bab ini berisi hasil tinjauan pustaka yang menjadi dasar analisa dan perancangan pada BAB III.
Bab ini secara garis besar berisi keamanan perangkat lunak dan pengujiannya,
tantangan pada pengujian perangkat lunak berbasis web, \emph{Domain-Specific Language}, serta BDD dan Gherkin.

% \begin{enumerate}
%     \item menunjukkan kepada pembaca adanya gap seperti pada rumusan masalah yang memang belum terselesaikan,
%     \item memberikan pemahaman yang secukupnya kepada pembaca tentang teori atau pekerjaan terkait yang terkait langsung dengan penyelesaian persoalan, serta
%     \item menyampaikan informasi apa saja yang sudah ditulis/dilaporkan oleh pihak lain (peneliti/Tugas Akhir/Tesis) tentang hasil penelitian/pekerjaan mereka yang sama atau mirip kaitannya dengan persoalan tugas akhir.
% \end{enumerate}

% \section{Dasar Teori}
%     \subsection{Subbab}
%     \begin{figure}[h]
%         \centering
%         \includegraphics[width=0.8\textwidth]{resources/chapter-2-infrastructure-diagram.png}
%         \caption{Contoh gambar}
%     \end{figure}

\section{Keamanan Perangkat Lunak}

Penggunaan komputer yang semakin hari semakin luas membuat perangkat lunak yang ada semakin besar dan rumit,
yang berarti juga bertambahnya masalah keamanan yang ada pada perangkat lunak tersebut.
Hal ini menyebabkan keamanan perangkat lunak menjadi hal yang semakin penting.

Keamanan perangkat lunak (\emph{software security}) adalah kriteria dimana perangkat lunak tetap bekerja
dengan benar walaupun diserang dengan niat jahat.
\emph{Security} berbeda dengan \emph{safety}
dimana security fokus terhadap kebenaran
perangkat lunak saat sedang dalam serangan yang dilakukan dengan sengaja, sedangkan \emph{safety} fokus terhadap
kebenaran perangkat lunak saat terjadi kegagalan baik pada tingkat perangkat lunak maupun perangkat keras.

Masalah keamanan perangkat lunak terjadi karena adanya celah atau kecacatan pada perangkat lunak yang dapat
dimanfaatkan oleh penyerang. Celah ini dapat berbentuk kekurangan bawaan pada bahasa pemrograman yang digunakan,
seperti penggunaan \texttt{gets()} pada bahasa C/C++ yang memiliki resiko \emph{buffer overflow},
hingga celah yang terjadi
karena kesalahan pada desain perangkat lunak tersebut. Skala pembuatan perangkat lunak yang semakin besar
dengan proses pengembangan yang melibatkan banyak orang menyebabkan tidak ada satu orang yang paham
cara kerja perangkat lunak secara keseluruhan.

Ada beberapa cara yang dapat dilakukan untuk menanggulangi masalah keamanan perangkat lunak,
namun pada saat ini perlindungan keamanan perangkat lunak dilakukan secara \emph{de facto},
yaitu dengan perlindungan yang diimplementasi setelah aplikasi selesai dikembangkan.
Perlindungan ini biasanya melindungi aplikasi dengan cara memperhatikan data yang masuk
ke dalam aplikasi tidak menimbulkan bahaya atau dapat menyebabkan masalah, pada dasarnya,
perlindungan jenis ini berdasar terhadap pencarian dan mengatasi celah pada aplikasi setelah ditemukan.
Namun, perlindungan perangkat lunak seharusnya mengidentifikasi dan mengatasi masalah dari
dalam perangkat lunak tersebut, sebagai contoh, walaupun ada baiknya mencoba menghadapi serangan
\emph{buffer overflow} dengan membaca \emph{traffic} yang masuk ke dalam aplikasi,
cara yang lebih bagus tentu saja memperbaiki perangkat lunak dari kodenya sehingga
tidak ada kemungkinan \emph{buffer overflow} (\cite{grawsec}).


\section{CWE dan \emph{Business Logic Error}}

\emph{Common Weakness Enumeration} (CWE) adalah daftar dan kategorisasi dari kelemahan keamanan
yang umum ditemukan pada perangkat lunak dan keras (\cite{cwe}). CWE dikelola oleh MITRE
yang merupakan organisasi non profit. CWE memungkinkan \emph{security engineer} untuk memiliki
bahasa umum untuk menyampaikan kelemahan keamanan yang ada. CWE juga mengeluarkan daftar 25
kelemahan keamanan yang paling banyak digunakan, daftar ini dapat digunakan oleh \emph{programmer}
dan \emph{tester} untuk membantu dalam pengembangan aplikasi.

\emph{Business Logic Error}(BLE) adalah kategori celah keamanan berasal dari kesalahan
yang biasanya memudahkan penyerang untuk memanipulasi logika bisnis aplikasi (\cite{ble_cwe}).
BLE biasanya sulit untuk ditemukan secara otomatis, karena mereka biasanya melibatkan
penggunaan fungsionalitas aplikasi dengan sah. Namun, banyak BLE dapat memiliki
pola-pola yang mirip dengan kelemahan implementasi dan detail yang sudah banyak dimengerti.

Klasifikasi dari BLE masih kurang dipelajari, walaupun eksploitasi dari BLE
sering terjadi di sistem nyata. Masih banyak perdebatan apakah BLE merepresentasikan
sebuah konsep baru, atau variasi dari konsep yang sudah dipahami.

Beberapa kategori dari BLE adalah:

\begin{enumerate}
      \item Melewati authentikasi dengan alur berbeda (CWE-288)
      \item Incorrect Behavior Order: Early Amplification (CWE-408)
      \item Melewati authorisasi dengan parameter dari user (CWE-408)
      \item Mekanisme pengembalian password yang lemah (CWE-640)
\end{enumerate}

Banyak BLE berorientasi terhadap proses bisnis, alur aplikasi, dan urutan perilaku,
yang dimana kelemahan-kelemahannya tidak banyak dipelajari di CWE.


\section{Pengujian Keamanan Perangkat Lunak}

Celah-celah keamanan yang ada pada perangkat lunak selalu menjadi risiko keamanan (\emph{security risk}).
Mengelola risiko keamanan ini menjadi seminimal mungkin adalah salah satu tugas praktisi keamanan perangkat lunak.
Dalam mengelola risiko ini dilakukan beberapa hal (\cite{grawsectest}), diantaranya:

\begin{itemize}
      \item Membuat kasus penyalahgunaan
      \item Membuat daftar kebutuhan keamanan
      \item Melakukan analisis risiko arsitektur
      \item Membuat perencanaan pengujian keamanan berbasis risiko
      \item Melakukan pengujian keamanan
      \item Melakukan pembersihan setelah terjadinya pelanggaran keamanan
\end{itemize}

Sistem keamanan bukanlah keamanan sistem. Walaupun fitur keamanan seperti \emph{cryptography},
\emph{access control}, dan lain lain memiliki peran penting dalam keamanan perangkat lunak,
keamanan itu sendiri adalah sifat dari sistem secara keseluruhan, bukan hanya dari mekanisme
dan fitur keamanannya. Sebuah \emph{buffer overflow} adalah masalah keamanan, baik itu terletak di dalam
fitur keamanan ataupun di dalam sebuah tampilan non-kritikal.
Karena itu dalam menguji keamanan perangkat lunak memiliki dua macam pendekatan (\cite{grawsec}):

\begin{enumerate}
      \item Menguji mekanisme keamanan untuk memastikan bahwa fungsionalitasnya telah diterapkan dengan baik
      \item Melakukan pengujian keamanan berbasis risiko berdasarkan pemahaman dan menyimulasikan pendekatan si penyerang sistem
\end{enumerate}

Banyak \emph{programmer} yang dengan salah mengira bahwa keamanan cukup hanya dengan mengimplementasikan dan
menggunakan fitur-fitur keamanan. Banyak penguji perangkat lunak yang ditugaskan untuk melakukan
pengujian keamanan melakukan kesalahan ini.

Seperti dalam pengujian lainnya, pengujian keamanan perangkat lunak terdiri dari memilih
siapa orang yang akan melakukan pengujian dan apa yang akan dilakukannya.
Dalam memilih orang ada dua kasus tergantung approach yang telah disebutkan,
pada kasus pertama dapat dilakukan oleh staff QA dengan cara pengujian
perangkat lunak seperti biasa untuk melakukan pengujian fungsional
fitur-fitur keamanan sesuai spesifikasi.
Namun pada kasus kedua, staff QA biasa akan kesulitan melaksanakan pengujian berbasis risiko
karena membutuhkan bidang keahlian tertentu.
Pertama, penguji harus dapat berpikir seperti penyerang sistem,
kedua, pengujian keamanan kadang tidak memberikan hasil yang berhubungan langsung dengan
celah keamanan yang ada, sehingga butuh keahlian untuk
menginterpretasi dan memahami hasil pengujian (\cite{grawsectest}).

Kedua, dalam memilih metode pengujian, ada dua metode yang dapat dilakukan.
Pertama dengan cara \emph{White-box} yang dilakukan dengan menganalisis dan memahami
kode serta desain dari program.
Cara ini cukup efektif dalam menemukan kesalahan pemrograman,
dalam beberapa kasus, pengujian ini dapat dilakukan oleh \emph{static analyzer}.
Cara kedua adalah pengujian \emph{Black-box} yang dilakukan dengan cara menguji program
yang sedang berjalan dengan berbagai macam masukan tanpa harus mengetahui masukan program.
Dalam pengujian keamanan, masukan buruk dapat dimasukkan dalam usaha untuk merusak program.
Kedua cara pengujian dapat mengungkapkan adanya risiko keamanan dan kemungkinan eksploitasi.
Masalah yang biasa terjadi dengan pengujian keamanan adalah terkadang organisasi atau perusahaan
tidak memiliki waktu dan sumberdaya untuk melakukan pengujian yang cukup.

Dalam melakukan pengujian keamanan, ada beberapa tantangan yang mungkin dihadapi (\cite{thompsonsectest}):
\begin{enumerate}
      \item Adanya efek samping

            Dalam melakukan pengujian keamanan dengan pendekatan menguji fungsionalitas perangkat lunak,
            biasanya diberikan sebuah masukan A dan diperiksa apakah perangkat lunak mengembalikan
            hasil B sesuai dengan spesifikasi.
            Namun yang kadang terlupakan bahwa aplikasi dapat memiliki efek samping yang dapat dimanfaatkan
            penyerang sebagai celah keamanan. Salah satu contohnya adalah perangkat utilitas
            RDISK pada Windows NT 4.0,  yang berfungsi untuk membuat \emph{Emergency Repair Disk}.
            Program ini pada umumnya berjalan baik sesuai spesifikasi, namun saat program berjalan,
            ia membuat sebuah file sementara yang dapat dibaca oleh siapa saja.
            Hal ini berarti pengguna tamu (\emph{guest}) dapat membaca isi file tersebut yang
            termasuk \emph{registry Windows} yang berisi pengaturan tentang sistem yang dapat dimanfaatkan penyerang.

      \item Keadaan Pengujian Keamanan Saat Ini

            Perusahaan yang menyediakan jasa pengujian keamanan biasanya memiliki daftar-daftar celah yang umum ada.
            Mereka biasanya hanya menggunakan daftar tersebut untuk membuat rencana pengujian.
            Cara seperti ini biasanya tidak akan dapat menemukan celah-celah keamanan yang baru.

      \item Ketidakamanan dan kegagalan aplikasi penunjang

            Perangkat lunak modern berjalan pada sistem yang saling bergantung satu sama lain,
            dimana satu aplikasi menggunakan puluhan \emph{library} dan berkomunikasi dengan
            beberapa komponen lainnya.
            Hal ini dapat menimbulkan dua masalah.
            Pertama, aplikasi dapat memiliki celah dari salah satu komponen yang ia gunakan.
            Kedua, sebuah komponen yang digunakan untuk menyediakan fungsionalitas keamanan
            dapat saja pada suatu saat rusak dan berhenti bekerja.

      \item Masukan tidak terkira dari pengguna

            Masukan dari pengguna adalah salah satu sumber celah yang paling umum dan paling mudah dieksploitasi.
            Beberapa contoh yang umum digunakan adalah masukan yang panjang, karakter spesial, dan nilai-nilai khusus.
            Salah satu contoh celah yang terjadi dari masukan pengguna ini adalah \emph{buffer overflow},
            yang memungkinkan penyerang menyisipkan kode pada masukan yang sangat panjang,
            hingga tidak bisa ditampung \emph{buffer} dan dijalankan oleh komputer.

      \item Ketidakamanan desain

            Banyak celah keamanan terjadi sejak perangkat lunak masih dalam tahap desain.
            Kadang celah tersebut tidak bisa langsung diketahui karena terjadi setelah semua
            bagian sistem selesai dirancang namun gabungan dari keseluruhan sistem tersebut
            menyebabkan adanya celah.
            Kadang celah juga terjadi pada test interface, yaitu bagian program yang sengaja
            disisipkan dan memberi celah untuk pengujian, namun tidak dihilangkan saat program akan dirilis.

      \item Ketidakamanan implementasi

            Walaupun spesifikasi perangkat lunak telah didesain sebaik mungkin dengan
            mempertimbangkan berbagai macam aspek keamanan,
            celah tetap dapat terjadi karena implementasi perangkat lunak yang tidak sempurna.

\end{enumerate}

\section{Tantangan Dalam Pengujian Aplikasi \emph{Web}}

Perangkat lunak berbasis web adalah salah satu jenis perangkat lunak paling umum pada saat ini.
Perangkat lunak ini menjadi tulang belakang dari komunikasi di dunia dan banyak hal-hal
yang membutuhkan keamanan tinggi menggunakan perangkat lunak berbasis web seperti perbankan.
Hal seperti menyebabkan perangkat lunak berbasis web menjadi salah satu target yang empuk untuk
dimanfaatkan celah dan kekurangannya. Sifat dari aplikasi web yang dinamis, kompleks, dan
selalu berubah-ubah membuat semakin mudahnya muncul celah baru
pada aplikasi web jika tidak diperhatikan (\cite{websecchal}).

Beberapa masalah umum yang ada pada perangkat lunak berbasis web adalah:
\begin{enumerate}
      \item Autentikasi: memastikan pengguna yang meminta data adalah benar pengguna tersebut
      \item Autorisasi: memastikan pengguna boleh melakukan hal yang dilakukannya.

      \item \emph{Cross-site scripting}:
            celah dimana penyerang dapat memasukkan kode jahat ke halaman web yang dijalankan di browser pengguna lain.

      \item \emph{SQL injection}:
            celah dimana disisipkannya kode jahat di dalam perintah SQL yang kemudian dijalankan oleh \emph{database}.

      \item \emph{Cross-site request forgery}:
            celah dimana sebuah \emph{website} dapat dieksploitasi untuk mengirimkan perintah palsu dari sebuah user.

      \item \emph{Malicious file execution}:
            aplikasi web menjalankan kode jahat yang berada di sebuah file bebas
\end{enumerate}

Beberapa tantangan dalam melakukan pengujian keamanan terhadap aplikasi web adalah (\cite{websecchal}):
\begin{enumerate}
      \item Butuhnya pengembangan kakas yang dapat mengotomatisasi pengujian aplikasi web.

      \item Pengembangan aplikasi web yang dinamis dan \emph{Rich Content} seperti \emph{Single-Page Application}
            mempersulit \emph{crawling} halaman web sehingga bisa saja ada state halaman yang
            tidak bisa dicapai oleh kakas pengujian.

      \item Bahasa pemrograman yang digunakan pada implementasi tidak memiliki fitur yang
            dapat memaksa penggunaan aturan keamanan yang dapat menyebabkan bahaya terhadap keamanan
            dan integritas data pengguna.
\end{enumerate}



\section{\emph{Behavior-Driven Development}}

\emph{Behavior-Driven Development}(BDD) adalah kerangka pengembangan dan pengujian perangkat lunak
yang mendorong percakapan dan contoh konkret untuk memberikan pemahaman bersama atas tingkah laku
perangkat (\cite{dan_bdd}). BDD adalah ekstensi dari kerangka TDD, dimana yang didefinisikan adalah
tingkah laku(\emph{behavior}) dari perangkat lunak, bukan kasus-kasus uji eksplisit.

% Dalam BDD, kasus uji dideskripsikan dengan skenario bagaimana perangkat lunak harus berperilaku terhadap
% masing masing fiturnya. Deskripsi ini dibuat dan dibahas oleh banyak pihak seperti \emph{programmer}
% , \emph{test engineer}, dan \emph{business analist}.
% Deskripsi skenario tersebut ditulis dalam langkah-langkah kecil (\emph{step}) dengan menggunakan
% bahasa yang simpel dan bisa dipahami oleh seluruh pihak.

Menurut (\cite{bdd_char}), BDD memiliki 6 karakteristik utama yaitu:
\begin{enumerate}
      \item \emph{Ubiquitous Language}

            \emph{Ubiquitous Language} (Bahasa Umum) adalah sebuah bahasa yang strukturnya berasal dari model domain dan
            mengandung istilah-istilah yang akan digunakan untuk mendeskripsikan perilaku suatu perangkat lunak.
            Bahasa umum yang didasari dari domain bisnis memungkinkan customer, bisnis, dan \emph{developer}
            saling berkomunikasi dengan jelas dan tanpa ambigu.

            BDD sendiri juga memiliki bahasa umumnya yang digunakan untuk mendeskripsikan fitur dan skenario
            perilaku perangkat lunak. Bahasa ini \emph{domain independent}.

      \item Proses Dekomposisi Iteratif

            Pada BDD analisis dimulai dengan identifikasi perilaku yang diharapkan dari sistem, yang lebih konkret
            dan mudah ditentukan. Lalu perilaku sistem akan diturunkan dari hasil bisnis yang seharusnya terjadi.
            Hasil bisnis itu kemudian diubah menjadi kumpulan fitur yang menyatakan apa saja yang harus ada
            agar hasil bisnis dihasilkan.
            Proses dekomposisi ini dilakukan secara iteratif, yang berarti tidak harus melakukan banyak analisi pada
            awalnya.

      \item Penjelasan \emph{User Story} dan Skenario dengan Simpel

            Pada BDD, biasanya deskripsi skenario, fitur, dan \emph{user story} ditulis dalam sebuah template tertentu
            dengan menggunakan bahasa simpel. Berbagai macam kakas BDD seperti JBehave, NBehave, SpecFlow, dan Cucumber
            menggunakan cara ini walaupun memiliki kata kunci berbeda, tetapi masih memiliki arti
            semantik yang sama.

      \item Pengujian Penerimaan Otomatis

            Pada BDD, skenario-skenario dari fitur yang telah sebelumnya dideskripsikan digunakan sebagai acuan
            untuk melakukan uji penerimaan (\emph{acceptance testing}) secara otomatis. \emph{Programmer} akan
            mulai dari salah satu skenario yang telah didefinisikan, yang kemudian dijadikan
            kode pengujian yang akan mengarahkan implementasi. Sebuah skenario terdiri dari langkah-langkah
            yang menggambarkan elemen-elemen yang ada dalam sebuah skenario.

      \item Kode Spesifikasi Berorientasi Perilaku yang Mudah Dibaca

            BDD menganjurkan bahwa kode seharusnya menjadi bagian dari dokumentasi sistem. Kode seharusnya
            mudah dibaca dan spesifikasi seharusnya menjadi bagian dari kode.

            StoryQ dan JSpec menyediakan API yang memungkinkan \emph{programmer} untuk mendeskripsikan
            \emph{user story} dan skenario sebagai kode. JBehave dan NBehave juga membantu untuk menulis
            skenario sebagai kode dengan menggunakan \emph{annotation}. Kebalikannya, Cucumber tidak
            berfokus kepada tingkat implementasi sehingga tidak memiliki karakteristik ini.

      \item \emph{Behaviour Driven} pada Fase Berbeda

            Karakteristik-karakteristik BDD yang telah dideskripsikan sebelumnya terjadi
            pada fase-fase yang berbeda selama dalam siklus pengembangan perangkat lunak.
            Pada fase rencana awal, perilaku perangkat lunak berhubungan dengan hasil bisnis.
            Pada fase analisis, hasil bisnis dipecah menjadi sekumpulan fitur yang melingkupi
            perilaku sistem.
            Pada fase implementasi, spesifikasi tersebut digunakan untuk mengarahkan implementasi
            dan pengujian otomatis. \emph{Programmer} dianjurkan untuk memikirkan perilaku dari
            komponen yang sedang mereka kembangkan dan interaksinya dengan komponen lain.

\end{enumerate}




\section{\emph{Domain-Specific Language}}

\emph{Domain-Specific Language} (DSL) adalah bahasa pemrograman berkemampuan terbatas yang
berfokus pada suatu domain tertentu (\cite{fowler_dsl}).

Dari definisi diatas, ada empat poin penting:

\begin{enumerate}
      \item DSL dapat digunakan untuk memerintahkan komputer. Seperti bahasa pemrograman lainnya,
            DSL haruslah bisa untuk dipahami manusia, tetapi masih mungkin diolah oleh komputer.

      \item DSL adalah bahasa pemrograman komputer. Hal ini berarti DSL harus terasa seperti
            bahasa yang dimana kemampuannya tidak hanya muncul dari masing-masing ekspresinya,
            tetapi juga saat ekspresi-ekspresi tersebut digabungkan.

      \item Bahasa pemrograman umum (\emph{General Purpose Programming Language}) memiliki banyak fitur.
            Hal ini membuatnya sangat berguna, namun menjadi susah untuk dipelajari dan digunakan.
            DSL dengan kemampuannya yang terbatas hanya memiliki fitur-fitur minimum yang dibutuhkan untuk domainnya.

      \item Sebuah bahasa dengan kemampuan terbatas hanya akan berguna jika ia memiliki
            fokus yang jelas terhadap sebuah domain kecil.
\end{enumerate}

DSL terbagi menjadi dua kategori, yaitu:

\begin{enumerate}
      \item \emph{External DSL}

            DSL eksternal adalah bahasa yang terpisah dari bahasa utama aplikasi.
            Biasanya, DSL eksternal memiliki syntaxnya sendiri, namun kadang dapat menggunakan
            bahasa lain seperti XML. Sebuah kode pada DSL eksternal biasanya akan diproses oleh
            aplikasi utama. Beberapa contoh DSL eksternal adalah Regex, SQL, awk, sed.

      \item \emph{Internal DSL}

            DSL internal adalah sebuah cara tertentu untuk menggunakan sebuah bahasa.
            Sebuah DSL internal ditulis dalam bahasa yang sama dengan bahasa utama aplikasi,
            namun hanya menggunakan sebagian fitur bahasa untuk mengurusi bagian kecil
            dari keseluruhan sistem. Salah satu bahasa yang memiliki banyak DSL internal adalah Ruby,
            karena struktur Ruby yang ekspresif memudahkan dibuatnya DSL.
            Web framework Rails yang ditulis dengan Ruby adalah salah satu contoh DSL.
\end{enumerate}

DSL adalah sebuah alat yang memiliki fokus yang jelas dan hanya mengurusi suatu
aspek kecil tertentu. Sebuah aplikasi bisa saja menggunakan banyak DSL untuk mengurusi
berbagai aspek sistemnya. Beberapa kelebihan menggunakan DSL adalah:

\begin{enumerate}
      \item Meningkatkan produktivitas

            Salah satu daya tarik utama dari DSL adalah ia menyediakan cara untuk menyampaikan
            sebuah maksud sebuah sistem dengan lebih jelas.
            Hal ini menyebabkan programmer lebih mudah memahami maksud dan
            tujuan sebuah kode dan sistem.

      \item Merepresentasikan pengetahuan domain dengan lebih baik

            DSL dapat didesain sedemikian mungkin untuk merepresentasikan dan mengabstraksikan
            suatu domain tertentu, sehingga
            bisa digunakan bukan hanya oleh \emph{programmer} saja, tetapi juga oleh ahli domain tersebut.
\end{enumerate}

Sementara kekurangan menggunakan DSL adalah:

\begin{enumerate}
      \item \emph{Language cacophony}

            Beberapa komplain yang sering didengar saat menggunakan DSL adalah \emph{language cacophony},
            dimana bahasa biasanya sulit untuk dipelajari, sehingga menggunakan banyak bahasa akan lebih
            sulit dari pada menggunakan satu bahasa. Kebutuhuan untuk mempelajari banyak bahasa menyebabkan
            sulit untuk mengerjakan proyek dan menambah orang baru kedalam proyek.

            Namun dari komplain ini, banyak orang yang berpikiran bahwa mempelajari sebuah DSL akan sesulit
            mempelajari bahasa pemrograman general biasa. Tetapi, DSL sebenarnya lebih mudah dipelajari
            karena keterbatasannya.

      \item Biaya pembuatan

            Seperti semua bagian dari program, DSL juga merupakan program yang harus dibuat dan dipelihara.
            Tentu saja hal ini dapat menambah biaya yang harus dikeluarkan. Biaya pembuatan DSL juga dapat
            lebih tinggi karena tim yang ada tidak terbiasa membuat DSL sehingga harus belajar lagi, yang
            juga dapat menambah biaya.

\end{enumerate}




\section{Cucumber dan Gherkin}

Cucumber adalah salah satu kakas yang mendukung kerangka BDD (\cite{cucumber_book}).
Cucumber memungkinkan \emph{programmer} untuk menulis spesifikasi
perilaku perangkat lunak dengan mudah dan kemudian dijalankan dalam pengujian,
spesifikasi ini ditulis dengan bahasa Gherkin.

Gherkin adalah sebuah DSL yang digunakan untuk mendeskripsikan fitur dan skenario
yang digunakan sebagai spesifikasi perilaku perangkat lunak. Gherkin ditulis
dengan bahasa manusia sehingga dapat dipahami oleh seluruh pihak, namun walaupun
ditulis dengan bahasa manusia, Gherkin dapat diolah dan digunakan oleh
komputer sebagai garis besar penjalanan pengujian otomatis.

\begin{lstlisting}[language=Gherkin]
Feature: Guess the word
  # The first example has two steps
  Scenario: Maker starts a game
    When the Maker starts a game
    Then the Maker waits for a Breaker to join

  # The second example has three steps
  Scenario: Breaker joins a game
    Given the Maker has started a game with the word "silky"
    When the Breaker joins the Maker game
    Then the Breaker must guess a word with 5 characters
\end{lstlisting}

Diatas adalah salah satu contoh kode Gherkin.
Test dalam Gherkin dibagi menjadi fitur-fitur. Tiap fitur tersebut akan dibagi menjadi
skenario, yang terdiri dari langkah-langkah.

% \subsection{\emph{Feature}}

% Fungsi dari \emph{Feature} adalah sebagai deskripsi level tinggi dari fitur perangkat lunak.
% \emph{Feature} juga berguna untuk mengelompokkan skenario yang berhubungan.
% Biasanya pada Cucumber, satu file hanya berisi satu \emph{Feature}.

% \subsection{\emph{Scenario / Example}}

% \emph{Scenario} adalah contoh konkret yang menggambarkan sebuah aturan bisnis yang terdiri
% dari langkah-langkah. \emph{Scenario} biasanya dideskripsikan dengan pola berupa:

% \begin{enumerate}
%       \item Nyatakan konteks awal (\emph{Given})
%       \item Nyatakan sebuah kejadian (\emph{When})
%       \item Nyatakan hasil yang diharapkan (\emph{Then})
% \end{enumerate}

\subsection{\emph{Step}}

\emph{Step} merupakan langkah kerja yang konkret.
Setiap \emph{step} dimulai dengan \texttt{Given}, \texttt{When}, \texttt{Then},
\texttt{And}, dan \texttt{But}.
Cucumber akan menjalankan tiap \emph{step} dalam sebuah \emph{scenario} secara berurutan.
Saat Cucumber akan menjalankan sebuah \emph{step}, ia akan mencari kode definisi step yang sesuai.
Misal jika sebuah \emph{step} ditulis sebagai \texttt{When the maker starts a game},
maka Cucumber akan mencari \emph{step definition} yang bernama
\texttt{the maker starts a game}.

\subsection{\emph{Step Definition}}

Cucumber memungkinkan terhubungnya antara definisi langkah pada Gherkin
dengan kode test nyata melalui nama pada \emph{step}.
Cucumber dapat digunakan dengan beberapa bahasa pemrograman berbeda.
\emph{Step definition} pada Cucumber didefinisikan menggunakan fungsi-fungsi simpel, seperti:

\begin{lstlisting}[language=java]
package com.example;
import io.cucumber.java.en.Given;

public class StepDefinitions {
    @Given("I have {int} cukes in my belly")
    public void i_have_n_cukes_in_my_belly(int cukes) {
        System.out.format("Cukes: %n\n", cukes);
    }
}
\end{lstlisting}

% \emph{Step definition} juga bisa dapat diberi variabel untuk meningkatkan \emph{code reuse}.

% \subsection{\emph{Scenario Outline}}

% Terkadang, ada beberapa \emph{scenario} yang ternyata memiliki kerangka yang sama,
% hanya saja berbeda pada bagian-bagian kecil seperti angka-angka. Beberapa skenario
% mirip ini dapat digabungkan menjadi satu skenario menggunakan \emph{scenario outline}.

% \begin{lstlisting}[language=gherkin]
%   Scenario Outline: eating
%   Given there are <start> cucumbers
%   When I eat <eat> cucumbers
%   Then I should have <left> cucumbers

%   Scenarios:
%     | start | eat | left |
%     |    12 |   5 |    7 |
%     |    20 |   5 |   15 |
% \end{lstlisting}