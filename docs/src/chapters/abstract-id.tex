\clearpage
\chapter*{ABSTRAK}
\addcontentsline{toc}{chapter}{Abstrak}

\begin{center}
  % \Large \bfseries ABSTRAK

  \Large \bfseries \MakeUppercase{\thetitle}
\end{center}

\begin{center}
  Oleh

  Ridho Pratama

  NIM : 13516032

\end{center}

%taruh abstrak bahasa indonesia di sini
\emph{Business Logic Error} (BLE) adalah jenis kelemahan keamanan dengan kode CWE-840.
BLE adalah kelemahan keamanan yang terjadi pada tingkat \emph{business logic}.
Keamanan ini sulit untuk diuji karena tidak dapat diotomasi seperti kelemahan keamanan lain.
Untuk menguji BLE secara otomatis membutuhkan pengetahuan tentang aksi dan state apa saja
yang mungkin terjadi dalam program. Pengetahuan ini terdapat pada deskripsi pengujian untuk
program yang menggunakan kerangka pengujian \emph{Behaviour-Driven Development} (BDD).
BDD adalah kerangka yang digunakan untuk pengujian program. Dengan BDD, pengujian dideskripsikan
dengan langkah-langkah bisnis. Langkah-langkah ini yang menjadi kumpulan pengetahuan yang dapat
dimanfaatkan untuk pengujian terhadap BLE, namun kakas BDD yang ada pada saat ini masih memiliki
banyak kekurangan untuk pengujian keamanan.

Pada tugas akhir ini berfokus untuk mengembangkan kakas yang dapat memanfaatkan pengetahuan yang
dimiliki BDD dengan menambahkan fitur-fitur yang mempermudah penggunaan BDD untuk pengujian keamanan,
khususnya BLE. Fitur yang ditambahkan adalah kemampuan representasi kegagalan, representasi variansi,
dan pengacakan skenario.

Hasil validasi menunjukkan bahwa fitur representasi kegagalan dan variansi dapat digunakan dengan baik
untuk mempermudah pengujian keamanan dan \emph{refactor} kasus uji, namun masih memiliki kekurangan
jika digunakan secara bersamaan dengan kombinasi tertentu.

Kata kunci: pengujian, keamanan, \emph{business logic}, BDD

\clearpage