\chapter{Penutup}

\section{Kesimpulan}

% Dalam pengembangan perangkat lunak, masih sulit dilakukan pengujian terhadap kelemahan keamanan berjenis \emph{Business Logic Error}.
% Hal ini terjadi karena kelemahan jenis ini membutuhkan pengoperasian perangkat lunak secara valid dengan langkah-langkah tertentu, dimana
% tiap langkah tersebut adalah langkah yang valid yang menyebabkan pengujian secara otomatis menjadi mustahil.

% BDD memberikan kumpulan pengetahuan langkah-langah \emph{business rule} yang mungkin terjadi terhadap perangkat lunak. Pengetahuan ini
% dapat dimanfaatkan untuk melakukan pengujian secara otomatis dengan membangkitkan skenario-skenario acak.
% BDD juga dapat ditambahkan fitur-fitur baru secara bahasa yang dapat mempermudah penguji untuk melakukan \emph{refactor} terhadap
% skenario pengujian, dan menambahkan skenario baru.

Dari pengerjaan tugas akhir ini dapat ditarik beberapa kesimpulan.

\begin{enumerate}
      \item Pengujian keamanan berbeda dengan pengujian fungsionalitas dimana pengujian fungsionalitas hanya menguji apakah fungsionalitas
            telah diimplementasikan dan berjalan dengan baik. Sementara pengujian keamanan menguji bahwa tidak ada celah-celah keamanan
            yang terbentuk. Pengujian keamanan sering tidak optimal karena \textit{programmer} biasanya hanya melakukan pengujian
            fungsionalitas saja.

      \item Pengujian sulit dilakukan walaupun telah menggunakan kakas sekalipun karena kakas pengujian masih memiliki banyak kekurangan
            yang menyebabkan programmer menjadi terdorong untuk tidak melakukan pengujian dengan baik. Untuk pengujian
            keamanan khususnya, kakas yang ada pada saat ini jauh berbeda dan tidak dapat bekerjasama dengan kakas
            yang digunakan pada pengujian fungsionalitas. Perbedaan ini menyebabkan adanya pengetahuan domain program
            yang dapat tidak tersampaikan saat dilakukan pengujian dari dua sisi.

      \item Hal-hal yang dibutuhkan pada kerangka pengujian keamanan, khususnya \textit{Business Logic Error}, adalah fitur
            yang dapat mempermudah pengujian keamanan. Fitur-fitur ini adalah representasi kegagalan, kemampuan menyatakan variansi,
            dan pengacakan skenario. Dua fitur pertama membantu dalam melakukan \textit{refactor} terhadap kode pengujian sehingga
            tidak terjadi duplikasi, dan fitur terakhir membantu penguji dalam menemukan kelemahan keamanan baru.
\end{enumerate}

\section{Saran}

Dari dua poin fitur pertama, pada pengujian ditemukan bahwa fitur \texttt{Fail Scenario} tidak dapat bekerjasama dengan fitur
\textit{Variabel Rejected}. Pada intinya saat kedua fitur ini digunakan akan menyebabkan negasi ganda terhadap skenario
dasar yang dihasilkan. Negasi ganda ini menyebabkan skenario yang dihasilkan tidak intuitif dan sulit untuk dilogikakan
oleh programmer. Solusi yang dapat dilakukan adalah dengan membatasi fitur variabel hanya dapat digunakan pada skenario
biasa, atau dengan melakukan rancangan kembali agar dapat menghasilkan desain yang lebih baik.

Pada implementasi, fitur pengacakan skenario menghasilkan skenario yang benar-benar acak tanpa ada aturan tertentu.
Hal ini menyebabkan skenario yang dihasilkan bernilai \textit{false negative}. Jumlah step pada program pengujian
yang sedikit menyebabkan tidak ditemukannya kelemahan terhadap program pengujian. Solusi yang dapat dilakukan
adalah dengan menggunakan algoritma pembangkitan skenario yang lebih kompleks sehingga dihasilkan skenario yang lebih efektif,
dan pengujian kakas juga dilakukan dengan program yang lebih besar dan komplek.

Untuk validasi kedepannya program yang akan diuji dapat dibuat lebih besar dan lebih kompleks sehingga dapat dilihat apakah
solusi yang didesain dapat bekerja dengan efektif untuk program yang besar.