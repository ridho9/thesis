\chapter{Strategi Implementasi}

\section{Parser}

Parser dibuat dengan menggunakan gaya \emph{Parsing Expression Grammar}(PEG)

Fitur seperti \emph{scenario outline} atau skenario dengan variable diolah pada bagian parser
menjadi skenario-skenario dasar.

Parser menghasilkan kumpulan skenario dalam bentuk data python yang mudah dibaca oleh runtime.

\section{Importer}

Importer berfungsi untuk membaca \emph{file} step descriptor yang ditulis dalam python.

Importer menghasilkan kumpulan fungsi \emph{step descriptor} yang akan digunakan untuk
menjalankan step-step yang telah didefinisikan dari skenario.

\section{Runtime}

Runtime berfungsi untuk menjalankan pengujian. Runtime menerima hasil dari parser dan importer,
mencocokkan step-step dalam skenario dengan fungsi \emph{step descriptor} yang cocok,
dan kemudian menjalankan skenario pengujian.

Runtime menghasilkan laporan penjalanan pengujian. Laporan ini berisi skenario apa saja
yang berhasil dan gagal. Laporan ini juga berisi penyebab kegagalan skenario dalam
bentuk catatan exception.

Fitur pembangkitan skenario acak diimplementasikan pada bagian runtime.