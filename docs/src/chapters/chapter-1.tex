\chapter{Pendahuluan}

\section{Latar Belakang}

Aplikasi web umum pada saat ini, seperti \emph{e-commerce}, umumnya berfokus pada mekanisme keamanan
seperti \emph{secure transfer protocol}, \emph{paramerter sanitization}, dan menggunakan bermacam skema
kriptografi. Para pengembang aplikasi tersebut lalu beranggapan dengan memberikan fitur keamanan seperti yang
disebutkan sudah cukup, padahal masih banyak kelemahan keamanan aplikasi terjadi pada
tingkat logika bisnis.

\emph{Business Logic Error} (CWE-840) adalah salah satu kelemahan
keamanan program yang disebabkan oleh kesalahan pada
tingkat implementasi logika bisnis.
Pengujian kelemahan ini tidak dapat diautomasi oleh kakas otomatis seperti \emph{scanner}
karena bergantung kepada domain dan bisnis aplikasi.
Kelemahan ini juga biasanya terlupakan atau tidak dilakukan karena pada siklus pengembangan aplikasi biasa,
yang diuji hanyalah kebutuhan fungsionalitas aplikasi apakah telah
terimplementasi dengan baik, sementara kelemahan-kelemahan yang mungkin ada tidak teruji.

Pada saat ini, pengujian keamanan biasanya dilakukan setelah pengembangan aplikasi selesai, dan pengujian
dilakukan dengan cara \emph{blackbox} yaitu aplikasi yang telah selesai diberikan bermacam-macam masukan.
Tetapi pengujian dengan cara \emph{blackbox} masih memiliki kelemahan dimana walaupun mungkin bisa menemukan
kelemahan keamanan yang sudah umum diketahui seperti \emph{injection}, cara ini masih jarang menemukan
kelemahan keamanan yang terjadi karena \emph{business logic error} yang biasanya membutuhkan
langkah-langkah yang sangat spesifik dan berbeda tiap aplikasinya.
%Ridho: Please define what is the business logic error... and how it is related to security which one is your main concern

Pengujian keamanan akan menjadi lebih efektif jika diintegrasikan ke dalam siklus pengembangan aplikasi,
seperti pengujian fungsionalitas.
Pada fungsionalitas, pengujian diintegrasikan ke dalam pengembangan dengan menggunakan kakas TDD dan BDD.
Kakas yang ada ini jika diikuti dengan baik bisa memberi jaminan bahwa fungsionalitas telah berjalan dengan baik.
Namun kakas ini belum bisa digunakan bersamaan untuk pengujian keamanan karena sifat dari pengujian keamanan itu sendiri.
Pengujian keamanan mengharuskan kita mencoba semua kemungkinan input yang pada normalnya tidak boleh
diterima oleh aplikasi.

\section{Rumusan Masalah}

Dari latar belakang tersebut, penulis kemudian merumuskan masalah yaitu:

\begin{enumerate}
    \item Apa yang menyebabkan pengujian keamanan berbeda dengan pengujian fungsionalitas?
    \item Kenapa pengujian sulit dilakukan dengan baik walaupun telah menggunakan kerangka pengujian?
    \item Apa saja hal yang dibutuhkan pada kerangka pengujian agar dapat mendukung pengujian keamanan?
          % \item Bagaimana cara mengintegrasikan pengujian keamanan pada tahap pengembangan aplikasi?
          % \item Bagaimana cara agar kakas pengujian yang ada pada saat ini dapat mengetahui input-input lain yang ada?
          % \item Bagaimana cara membuat keamanan mudah diuji pada tahap pengembangan?%dukungan kakas seperti apa yang ....
\end{enumerate}

\section{Tujuan}

Subbab sebelum ini telah menjelaskan latar belakang dan rumusan masalah tugas akhir ini.
Karena itu, tujuan dari tugas akhir ini adalah membangun kakas pengujian aplikasi
dengan kerangka BDD atau TDD, dimana kakas dapat juga dapat melakukan pengujian
keamanan dari \emph{business logic error} dengan mudah.

\section{Batasan Masalah}

Untuk mencapai tujuan yang telah dijelaskan pada subbab sebelumnya, kakas pengujian difokuskan pada hal berikut:

\begin{enumerate}
    \item Untuk menguji ancaman keamanan yang disebabkan oleh \emph{business logic error}
    \item Kerangka pengujian yang digunakan sebagai acuan adalah kerangka pengujian BDD Gherkin
\end{enumerate}

\section{Metodologi}

Metodologi yang digunakan pada pengerjaan tugas akhir ini antara lain:

\begin{enumerate}
    \item Studi Literatur

          Pada tahap ini dilakukan penggalian pengetahuan melalui literatur-literatur yang terkait
          dengan topik tugas akhir seperti \textit{paper, journal} dan buku.

    \item Eksplorasi kebutuhan

          Pada tahap ini dilakukan analisa penyebab terjadinya \emph{business logic error}.
          Lalu dilakukan penentuan fitur yang dibutuhkan untuk bahasa pengujian.

    \item Desain

          Pada tahap ini dilakukan desain bahasa dan arsitektur kakas yang akan dibuat.

    \item Implementasi

          Pada tahap ini dilakukan implementasi kakas yang telah didesain pada bagian sebelumnya.

    \item Evaluasi

          Pada taham ini dilakukan evaluasi terhadap kakas yang telah dibuat dan fitur yang dirancang
          dengan cara melakukan pengujian terhadap program nyata.
\end{enumerate}


\section{Sistematika Penulisan}

Tugas akhir ini ditulis dalam 5 bab. Penjelasan masing-masing bab adalah sebagai berikut.

\begin{enumerate}
    \item Bab pendahuluan berisi latar belakang, rumusan masalah, tujuan, batasan masalah, metodologi, dan sistematika penulisan.

    \item Bab tinjauan pustaka berisi teori-teori pendukung yang digunakan pada tugas akhir ini seperti pengujian perangkat lunak,
          pengujian keamanan, \textit{Business Logic Error}, \textit{Behaviour Driven Development}, dan \textit{Domain-Specific Language}

    \item Bab analisis dan perancangan berisi analisis permasalahan, kebutuhan DSL, rancangan DSL, dan desain arsitektur.

    \item Bab implementasi membahas implementasi dan pengujian kakas yang terdiri dari detail implementasi, tujuan pengujian, hasil dan analisis pengujian.

    \item Bab penutup berisi kesimpulan dan saran terhadap tugas akhir.
\end{enumerate}