\chapter{Pendahuluan}

\section{Latar Belakang}

Aplikasi web umum pada saat ini, seperti \textit{e-commerce}, umumnya berfokus pada mekanisme keamanan
seperti \textit{secure transfer protocol}, \textit{paramerter sanitization}, dan menggunakan bermacam skema
kriptografi. Para pengembang aplikasi tersebut lalu beranggapan dengan memberikan fitur keamanan seperti yang
disebutkan sudah cukup, padahal masih banyak kelemahan keamanan aplikasi terjadi pada
tingkat logika bisnis.

\textit{Business Logic Error} (CWE-840) adalah salah satu kelemahan keamanan program yang disebabkan oleh kesalahan pada
tingkat implementasi logika bisnis.
Pengujian kelemahan ini tidak dapat diautomasi oleh kakas otomatis seperti \textit{scanner}
karena bergantung kepada domain dan bisnis aplikasi.
Kelemahan ini juga biasanya terlupakan atau tidak dilakukan karena pada siklus pengembangan aplikasi biasa,
yang diuji hanyalah kebutuhan fungsionalitas aplikasi apakah telah
terimplementasi dengan baik, sementara kelemahan-kelemahan yang mungkin ada tidak teruji.

Pada saat ini, pengujian keamanan biasanya dilakukan setelah pengembangan aplikasi selesai, dan pengujian
dilakukan dengan cara \textit{blackbox} yaitu aplikasi yang telah selesai diberikan bermacam-macam masukan.
Tetapi pengujian dengan cara \textit{blackbox} masih memiliki kelemahan dimana walaupun mungkin bisa menemukan
kelemahan keamanan yang sudah umum diketahui seperti \textit{injection}, cara ini masih jarang menemukan
kelemahan keamanan yang terjadi karena \textit{business logic error} yang biasanya membutuhkan
langkah-langkah yang sangat spesifik dan berbeda tiap aplikasinya.

Pengujian keamanan akan menjadi lebih efektif jika diintegrasikan kedalam siklus pengembangan aplikasi.

\section{Rumusan Masalah}

Dari latar belakang tersebut, penulis kemudian merumuskan masalah yaitu:

\begin{enumerate}
    \item Kenapa \textit{business logic error} terjadi?
    \item Kenapa pengujian sulit dilakukan dengan baik walaupun telah menggunakan kerangka pengujian?
    \item Apa yang menyebabkan pengujian keamanan berbeda dengan pengujian fungsionalitas?
    \item Bagaimana cara membuat keamanan mudah diuji pada tahap pengembangan?
\end{enumerate}

\section{Tujuan}

Tuliskan tujuan utama dan/atau tujuan detil yang akan dicapai dalam pelaksanaan tugas akhir. Fokuskan pada hasil akhir yang ingin diperoleh setelah tugas akhir diselesaikan, terkait dengan penyelesaian persoalan pada rumusan masalah. Penting untuk diperhatikan bahwa tujuan yang dideskripsikan pada subbab ini akan dipertanggungjawabkan di akhir pelaksanaan tugas akhir apakah tercapai atau tidak.

Subbab sebelum ini telah menjelaskan latar belakang dan rumusan masalah tugas akhir ini.
Karena itu, tujuan dari tugas akhir ini adalah membangun ????? 

\section{Batasan Masalah}

Untuk mencapai tujuan yang telah dijelaskan pada subbab sebelumnya, ????? difokuskan pada hal berikut:

\begin{enumerate}
    \item 
\end{enumerate}

\section{Metodologi}

Metodologi yang digunakan pada pengerjaan tugas akhir ini antara lain:

\begin{enumerate}
    \item 
\end{enumerate}