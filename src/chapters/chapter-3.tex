\chapter{Analisis dan Perancangan}

\section{Perbedaan Pengujian Fungsionalitas dan \emph{Business Logic Error}}

Pengujian untuk fungsionalitas dan untuk keamanan memiliki perbedaan. 
Pada pengujian fungsionalitas, yang dibutuhkan hanyalah untuk menguji apakah
fungsionalitas telah terimplementasikan dengan benar.
Namun pada pengembangan aplikasi terkadang secara tidak sengaja ada fungsionalitas
tambahan yang bisa menjadi celah keamanan yang sering terlupakan untuk diuji.
Salah satu macam dari celah ini adalah \emph{business logic error}.

Pengujian untuk \emph{business logic error} dilakukan dengan memperhatikan tiap-tiap fungsionalitas
yang diimplementasikan dalam program dan menguji kemungkinan variasi keadaan pada
fungsionalitas tersebut. Salah satu contohnya adalah pada aplikasi \emph{e-commerce} yang memiliki
fitur keranjang. Pada fungsionalitas "bisa menambahkan barang ke dalam keranjang", dapat memiliki
prasyarat bahwa user hanya bisa memasukkan barang ke dalam keranjang jika sudah login, dan skenario
ini lah yang biasa diuji pada pengujian fungsionalitasnya. Namun untuk \emph{business logic error} 
terjadi saat asumsi-asumsi normal itu tidak berlaku seperti, bagaimana jika user belum login,
bagaimana jika user memasukkan -1 barang, dan lain-lain.

Salah satu cara pengujian fungsional dapat diintegrasikan ke dalam siklus pengembangan aplikasi
adalah dengan menggunakan kerangka pengujian BDD atau TDD. Kerangka ini mengharuskan \emph{programmer}
untuk menulis spesifikasi dan kasus uji sebelum mulai menulis kode.
Salah satu dari kerangka ini adalah Cucumber BDD, beserta bahasa yang ia gunakan untuk mendeskripsikan
kasus uji, yaitu Gherkin.

Gherkin adalah bahasa yang digunakan untuk mendeskripsikan spesifikasi fungsionalitas
aplikasi. Spesifikasi ini ditulis per-skenario, dimana tiap-tiap skenario dibagi menjadi
langkah-langkah simpel yang menjelaskan skenario tersebut. Dari Gherkin ini kemudian
digunakan oleh Cucumber untuk menjalankan pengujian.

Gherkin jika digunakan dengan baik dapat membantu implementasi dan
pengujian fungsionalitas aplikasi. Tetapi, Gherkin masih belum bisa digunakan untuk melakukan
pengujian \emph{business logic error} dengan mudah. Namun ada dua hal yang dapat ditambahkan
ke dalam Gherkin untuk mempermudah pengujian \emph{business logic error}, yaitu kemampuan untuk
mengecek kegagalan, dan kemampuan untuk mendefinisikan banyak state/value dengan mudah.

\section{Failure}

Kekurangan pertama Gherkin yang akan dibahas adalah ketidakmampuan untuk
menyatakan suatu skenario harus gagal.
Hal ini terjadi karena desain Gherkin yang bertujuan untuk menguji fungsionalitas saja,
sehingga setiap step Gherkin dianggap benar jika tidak ada \emph{error}/\emph{exception}
dari fungsi implementasi stepnya.
Namun untuk pengujian \emph{business logic error}, dibutuhkan kemampuan untuk menyatakan
bahwa suatu skenario atau stepnya harus gagal, dan pengujian dianggap tidak lolos
jika skenario tersebut tidak gagal (dalam bentuk \emph{error/exception}).

Dengan kemampuan untuk mendefinisikan bahwa suatu skenario harus gagal,
\emph{programmer} dimudahkan untuk membuat lebih banyak test case
yang menguji \emph{business logic error}.

\section{State Domain}

Pada saat ini, Gherkin memiliki fitur \emph{scenario outline}
untuk mendefinisikan sebuah \emph{template} skenario,
didalamnya dapat digunakan \emph{placeholder} variabel yang dapat didefinisikan
dalam sebuah tabel.
Menggunakan \emph{scenario outline} memungkinkan \emph{programmer} untuk
mendefinisikan banyak skenario yang mirip dengan nilai berbeda dengan mudah.